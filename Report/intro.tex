\documentclass[main.tex]{subfiles}
\begin{document}
  \begin{abstract}
    A carpooling recommendation system can alleviate the problems of traffic
    congestion and environmental pollution effectively in big cities. The system
    partitions commuters travelling to the same destination based on vehicle
    capacities, their periodic travel times and their mutual preferences to
    share rides with each other. If the set of drivers is known in advance, then
    for any vehicle capacity, the problem is equivalent to the assignment
    problem in bipartite graphs. Otherwise, when we do not know in advance who
    will drive their vehicle and who will be a passenger, the problem is
    NP-hard. The road distances between the set of commuters and from commuters
    to the major roads are modeled as a directed graph which acts as a
    simplified map of the city layout. In this project, we aim to develop a
    proof of concept to solve a restricted version of the problem limited to a
    single common destination.
  \end{abstract}

\section{Introduction}
The carpooling recommendation has become one of the major problems of traffic
congestion and environmental pollution in big cities and hence to sort that out,
considering various graph based algorithms we have tried to come up with a novel
method to resolve the same.

A lot of car sharing serviced need strong algorithms giving them the maximum
output rates. Maximum output rates means utilizing maximum number of seats per
vehicle and in the shortest path travelled. Keeping the paper "Theory and
Practice in Large Carpooling Problems" by Irith Ben-Arroyo Hartman as our mentor
paper we proposed our own algorithm to achieve the same.  To work on this
algorithm, the first requirement is obtaining all the points or nodes for the
graph; this is done by obtaining the coordinates of the places where people are
expecting the carpooling service to be obtained.

There are millions of ways and parameters that had to be initially kept to come
up with this algorithm. Using the osmnx library we were successful in importing
the map of a particular region. The user interface is easy to work with and the
points can be easily selected by clicking on the specific region on the map.
After the cordinates of all the people interested in carpooling have been
selected we select one destination node. This will be further illustrated in an
example in the next section.

Using the function call networkx() which works like the Dijkstra algortihm, it
computes the shortest paths between all the points and gives us a complete
graph.  After obtaining the complete graph, we now have the distances from all
points to all other points. This information can be used to calculate the most
optimized path such that every party is benefited.
Initial contraints considered:
\begin{itemize}
  \item Vehicle capacity - 2 people
  \item Destination      - 1 common destination
  \item Threshold value that a person is willing to travel extra to pick up
    another one - 200m
\end{itemize}

After the first graph was obtained, the main purpose of the entire thing is to
find maximal matching. In order to do so, using the triangle inequality, so long
the distance between the source and the destination when including a new node in
between does not exceed the threshold value, an edge is drafted between those
two nodes. This way it checks for all the nodes which can take the other nodes
along whilst reaching the destination.
This way our final graph is obtained from which we are interested in obtaining
the maximal distance. Using the function call to the maximal function we
eventually get the set of co-ordinates for the calculation of the optimal
carpooling methodology.
The following algorithms have been used for the same
\begin{itemize}
  \item Dijsktra shortest path
  \item Filtering algorithm based on Floyd-Warshall algorithm
  \item Maximal matching usign ``blossom'' and ``primal-dual'' method.
\end{itemize}
\end{document}
