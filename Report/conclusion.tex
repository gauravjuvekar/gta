\documentclass[main.tex]{subfiles}
\begin{document}
\section{Conclusion}
Despite obtaining a set of maximal matchings, there were a lot of constraints
that were considered initially. To make the algorithm work smoothly. After the
execution of the algortihm we can easily obtain a set of maximal points that
give us the best way two people can carpool in the most optimum way, which again
depends on the threshold value which for exerimentation purposes we have kept as
one.
\section{Future Scope}
However, there is still a lot more potential in this area.  First, if infact
keeping the vehicle capacity as 2 fixed, the user should be given the authority
of selecting the vehicle capacity, as this brings a lot of changes to the final
algorithm.  The other way this algorithm can be extented is by not keeping one
common destination for all the nodes. Keeping this factor, the algorithm needs
to be tweaked accordingly.  This is just a small little solution to the huge
problem existing for finding the most optimum and the efficient means of
carpooling to save petrol and reduce pollution.

\section{References}
\begin{itemize}
  \item ``Theory and Practice in Large Carpooling Problems'', Irith Ben-Arroyo
    Hartman et al. (ANT-2014)
  \item ``Efficient Algorithms for Finding Maximum Matching in Graphs'', Zvi
    Galil, ACM Computing Surveys, 1986. (networkx implementation)
  \item ``Floyd-Warshall algorithm''
  \item \texttt{osmnx} and \texttt{networkx} documentation
\end{itemize}

\end{document}
